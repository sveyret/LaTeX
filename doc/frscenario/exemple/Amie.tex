\documentclass{frscenario}

\Titre{Une amie pas comme les autres}
\Auteur{Marie A.}
\Contact{https://www.atramenta.net/lire/une-amie-pas-comme-les-autres/53613\\
Œuvre sous licence Creative Commons by-nc-nd 3.0\\
En lecture libre sur Atramenta.net\\
Modifié pour servir d'exemple à la classe \LaTeX{} frscenario}

\begin{document}

\SceneExt{jour}{Un parc}

Léa 15 ans, se promène dans un parc. Elle est angoissée. Devant elle, une jeune fille au style extravagant danse sur une chanson de Michael Jackson « They don’t care about us »

\— Léa 1

Bonjour. Je me suis perdue. Vous pouvez m’indiquer le chemin qui mène à l’entrée du village ?

\— Léa 2

Hello
\did{elle lui donne une tape amicale}
Bien sûr que je vais te guider. Je me présente, Léa !

\— Léa 1

\did{surprise}
Moi aussi, je m’appelle Léa.

\— Léa 2

Je sais… Je te connais depuis longtemps…

\— Léa 1

\did{étonnée}
Comment pouvez-vous me connaître depuis ma naissance ? On vient juste de se rencontrer.

\— Léa 2

Pourtant je t’accompagne partout. Je connais toute ta vie. Je connais tous tes secrets. Je nage à travers tes pensées. Je sais -{}-

\— Léa 1

\did{En lui coupant la parole}

Bon vous êtes qui ! Vous me faites perdre mon temps !

\— Léa 2

Je suis ta conscience.

\— Léa 1

Ma conscience ?

\did{À elle-même}
C’est un rêve, tu vas te réveiller !

\——

Léa 2 la pince… Léa 1 sursaute.

\— Léa 1

Pourquoi, je me réveille pas !
\did{un temps}
Si je ne rêve pas c’est que je suis morte…

\— Léa 2

T’es pas morte… sinon tu serais transparente.

\— Léa 1

\did{Toujours à elle-même}
Pourtant, je n’ai rien bu hier soir… C’est pas normal.

\— Léa 2

Non plus !

\— Léa 1

Je suis paranoïaque alors !

\— Léa 2

Non ! C’est normal d’avoir une voix dans sa tête…

\— Léa 1

Épargne-moi tes débats philosophiques !

\——

Léa 1 part en courant, Léa 2 la rattrape. Elle la force à la suivre.

\— Léa 2

T’es pas parano, on a tous une conscience…

D’ailleurs ta conscience te dit d’aller réviser tes partiels, sinon tu ne vas pas valider ton année. Et aussi de mettre une écharpe et demander plus de soupe.

\— Léa 1

On dirait ma mère. Tu peux me laisser tranquille ?

\did{D’un coup en colère}
Pars de ma tête ! Maintenant ! Tout de suite ! Pars ! Sinon j’appelle la police.

\— Léa 2

\did{en désignant une route}
Du calme ! Regarde il est là le chemin.

\— Léa 1

Merci…

\— Léa 2

Hé, ça te dirait d’être ma pote ?

\— Léa 1

Non, mais ça va pas ? Moi, amie avec toi !!!!!!!!!!!!!!!

\——

Elle s’éloigne. Léa 2 la regarde en soupirant

\— Léa 2

C’est pas gagné…

\——

Elle s’éloigne à son tour en sautillant, tout écoutant sa musique.

\SceneInt{nuit}{Chambre de Léa 1}

Léa 1 s’agite dans son lit. Elle allume la lumière et regarde sa convocation pour une audition de danse. Léa 2 apparaît brusquement. Comme dans la première scène, elle est habillée dans un style décalé.

\— Léa 2

Arrête de penser à ça
\did{Elle lui prend le papier des mains}
On dirait un poisson hors de l’eau. Tu es une superbe danseuse. Tu vas réussir ton audition !

\— Léa 1

Encore toi ? Tu peux me laisser tranquille !

\— Léa 2

Tu es une fille géniale, Léa ! Pense à tes qualités, pas tes défauts. Tu es toujours en train de te dire
\did{en l’imitant de manière exagérée}
je suis bête, je suis moche, j’y arriverai pas… Tu seras toujours agacée…

\— Léa 1

C’est toi qui m’agaces !

\— Léa 2

Si tu penses tout le temps à tes défauts, tu auras peur de ce que les gens pensent de toi. Mais si tu penses à tes qualités ça ira beaucoup mieux ! Tu verras si t’es ma pote, tu seras plus heureuse.

\— Léa 1

J’ai déjà des potes ! Pas besoin de toi !

\— La Mère (off)

Ça va Léa ? À qui tu parles ?

\— Léa 1

À une fille complètement dingue qui vit dans ma tête !

\— Léa 2

Si t’es pas ami avec ta conscience et tes pensées, tu seras jamais en paix… ça sera toujours la guerre éternelle. La guerre avec soi-même est la plus violente et la plus difficile à mener de toutes les guerres. Nous en sommes les durs dictateurs. On s’auto-censure, on s’auto-détruit. Alors si tu veux signer l’armistice, c’est maintenant.

\— Léa 1

\did{Pour être tranquille}
Je vais réfléchir ! Bonne nuit !

\— Léa 2

Merde pour ton audition
\did{À elle-même}
Qu’est-ce qu’elle est indécise !

\——

\SceneExt{jour}{Devant le conservatoire}

Léa 1 sort du conservatoire en souriant. Léa 2, l’attend, assise sur les marches.

\— Léa 2

Bravo pour ton audition !

\— Léa 1

Merci, pour tes conseils. Ça m’a beaucoup aidée. Je n’ai eu que des éloges du jury. Ils ont été surpris par ma grâce et mon enthousiasme. Ils ont apprécié ma qualité de présence…

\— Léa 2

Alors tu veux enfin être ma pote ?

\— Léa 1

\did{un temps, elle sourit}
C’est d’accord !

\— Léa 2

Super ! Tope-là ma pote, toi et moi on va devenir inséparables, on va gagner la guerre du pétrole, on va devenir les directeurs de la NASA, on gravira le Mont-Blanc et on ira visiter l’épave du Titanic !

\— Léa 1

Peut-être pas quand même !

\— Léa 2

C’est toi qui le penses !

\——

Elle s’éloigne en sautillant et chantant « Envole-moi » Peu à peu Léa 2 disparaît…

\transition{Fondu au noir.}

\Generique

L'œuvre initiale ne définit ni transition ni générique. Ceci a été ajouté pour l'exemple.

\end{document}

