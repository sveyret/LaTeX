\documentclass{frscenario}

\Titre{Une amie pas comme les autres}
\Auteur{Marie A.}
\Contact{https://www.atramenta.net/lire/une-amie-pas-comme-les-autres/53613\\
Œuvre sous licence Creative Commons by-nc-nd 3.0\\
En lecture libre sur Atramenta.net\\
Modifié pour servir d'exemple à la classe \LaTeX{} frscenario}

\begin{document}

\SceneExt{jour}{Un parc}

Léa 15 ans, se promène dans un parc. Elle est angoissée. Devant elle, une jeune fille au style extravagant danse sur une chanson de Michael Jackson « They don’t care about us »

\— Léa 1

Bonjour. Je me suis perdue. Vous pouvez m’indiquer le chemin qui mène à l’entrée du village ?

\— Léa 2

Hello
\did{elle lui donne une tape amicale}
Bien sûr que je vais te guider. Je me présente, Léa !

\— Léa 1

\did{surprise}
Moi aussi, je m’appelle Léa.

\— Léa 2

Je sais… Je te connais depuis longtemps…

\— Léa 1

\did{étonnée}
Comment pouvez-vous me connaître depuis ma naissance ? On vient juste de se rencontrer.

\— Léa 2

Pourtant je t’accompagne partout. Je connais toute ta vie. Je connais tous tes secrets. Je nage à travers tes pensées. Je sais -{}-

\— Léa 1

\did{En lui coupant la parole}

Bon vous êtes qui ! Vous me faites perdre mon temps !

\— Léa 2

Je suis ta conscience.

\— Léa 1

Ma conscience ?

\did{À elle-même}
C’est un rêve, tu vas te réveiller !

\——

Léa 2 la pince… Léa 1 sursaute.

\— Léa 1

Pourquoi, je me réveille pas !
\did{un temps}
Si je ne rêve pas c’est que je suis morte…

\— Léa 2

T’es pas morte… sinon tu serais transparente.

\— Léa 1

\did{Toujours à elle-même}
Pourtant, je n’ai rien bu hier soir… C’est pas normal.

\— Léa 2

Non plus !

\— Léa 1

Je suis paranoïaque alors !

\— Léa 2

Non ! C’est normal d’avoir une voix dans sa tête…

\— Léa 1

Épargne-moi tes débats philosophiques !

\——

Léa 1 part en courant, Léa 2 la rattrape. Elle la force à la suivre.

\— Léa 2

T’es pas parano, on a tous une conscience…

D’ailleurs ta conscience te dit d’aller réviser tes partiels, sinon tu ne vas pas valider ton année. Et aussi de mettre une écharpe et demander plus de soupe.

\— Léa 1

On dirait ma mère. Tu peux me laisser tranquille ?

\did{D’un coup en colère}
Pars de ma tête ! Maintenant ! Tout de suite ! Pars ! Sinon j’appelle la police.

\— Léa 2

\did{en désignant une route}
Du calme ! Regarde il est là le chemin.

\— Léa 1

Merci…

\— Léa 2

Hé, ça te dirait d’être ma pote ?

\— Léa 1

Non, mais ça va pas ? Moi, amie avec toi !!!!!!!!!!!!!!!

\——

Elle s’éloigne. Léa 2 la regarde en soupirant

\— Léa 2

C’est pas gagné…

\——

Elle s’éloigne à son tour en sautillant, tout écoutant sa musique.

\SceneInt{nuit}{Chambre de Léa 1}

Léa 1 s’agite dans son lit. Elle allume la lumière et regarde sa convocation pour une audition de danse. Léa 2 apparaît brusquement. Comme dans la première scène, elle est habillée dans un style décalé.

\— Léa 2

Arrête de penser à ça
\did{Elle lui prend le papier des mains}
On dirait un poisson hors de l’eau. Tu es une superbe danseuse. Tu vas réussir ton audition !

\— Léa 1

Encore toi ? Tu peux me laisser tranquille !

\— Léa 2

Tu es une fille géniale, Léa ! Pense à tes qualités, pas tes défauts. Tu es toujours en train de te dire
\did{en l’imitant de manière exagérée}
je suis bête, je suis moche, j’y arriverai pas… Tu seras toujours agacée…

\— Léa 1

C’est toi qui m’agaces !

\— Léa 2

Si tu penses tout le temps à tes défauts, tu auras peur de ce que les gens pensent de toi. Mais si tu penses à tes qualités ça ira beaucoup mieux ! Tu verras si t’es ma pote, tu seras plus heureuse.

\— Léa 1

J’ai déjà des potes ! Pas besoin de toi !

\— La Mère (off)

Ça va Léa ? À qui tu parles ?

\— Léa 1

À une fille complètement dingue qui vit dans ma tête !

\— Léa 2

Si t’es pas ami avec ta conscience et tes pensées, tu seras jamais en paix… ça sera toujours la guerre éternelle. La guerre avec soi-même est la plus violente et la plus difficile à mener de toutes les guerres. Nous en sommes les durs dictateurs. On s’auto-censure, on s’auto-détruit. Alors si tu veux signer l’armistice, c’est maintenant.

\— Léa 1

\did{Pour être tranquille}
Je vais réfléchir ! Bonne nuit !

\— Léa 2

Merde pour ton audition
\did{À elle-même}
Qu’est-ce qu’elle est indécise !

\——

\SceneExt{jour}{Devant le conservatoire}

Léa 1 sort du conservatoire en souriant. Léa 2, l’attend, assise sur les marches.

\— Léa 2

Bravo pour ton audition !

\— Léa 1

Merci, pour tes conseils. Ça m’a beaucoup aidée. Je n’ai eu que des éloges du jury. Ils ont été surpris par ma grâce et mon enthousiasme. Ils ont apprécié ma qualité de présence…

\— Léa 2

Alors tu veux enfin être ma pote ?

\— Léa 1

\did{un temps, elle sourit}
C’est d’accord !

\— Léa 2

Super ! Tope-là ma pote, toi et moi on va devenir inséparables, on va gagner la guerre du pétrole, on va devenir les directeurs de la NASA, on gravira le Mont-Blanc et on ira visiter l’épave du Titanic !

\— Léa 1

Peut-être pas quand même !

\— Léa 2

C’est toi qui le penses !

\——

Elle s’éloigne en sautillant et chantant « Envole-moi » Peu à peu Léa 2 disparaît…

\SceneExtInt{nuit}{Devant la porte d'une maison}

Tout ce qui suit a été ajouté au scénario originel pour servir d'exemple à l'utilisation de la classe.

La caméra est devant la porte ouverte d'une maison. Elle entre.

La suite est un extrait du « Mystère de la chambre jaune » de « Gaston Leroux » dont l'objectif est de faire un très long dialogue.

\SceneInt{nuit}{Tribunal}

\— Frédéric Larsan

Monsieur le président, il serait intéressant d’entendre M. Joseph Rouletabille ; d’autant plus intéressant qu’il n’est pas de mon avis.

\——

Un murmure d’approbation accueille cette parole du policier. Il accepte le duel en beau joueur. La joute promet d’être curieuse entre ces deux intelligences qui se sont acharnées au même tragique problème et qui sont arrivées à deux solutions différentes.

Comme le président se tait :

\— Frédéric Larsan

Ainsi nous sommes d’accord pour le coup de couteau au cœur qui a été donné au garde par l’assassin de Mlle Stangerson ; mais, puisque nous ne sommes plus d’accord sur la question de la fuite de l’assassin, « dans le bout de cour », il serait curieux de savoir comment M. Rouletabille explique cette fuite.

\— Joseph Rouletabille

Évidemment, ce serait curieux !

\——

Toute la salle part encore à rire.

\— Le président

Si un pareil fait se renouvelle, je n’hésiterai pas à mettre à exécution ma menace de faire évacuer la salle.

Vraiment, dans une affaire comme celle-là, je ne vois pas ce qui peut prêter à rire.

\— Joseph Rouletabille

Moi non plus !

\——

Des gens, devant la caméra, s’enfoncent leur mouchoir dans la bouche pour ne pas éclater…

\— Le président

Allons, vous avez entendu, jeune homme, ce que vient de dire M. Frédéric Larsan. Comment, selon vous, l’assassin s’est-il enfui du « bout de cour » ?

\——

Rouletabille regarde Mme Mathieu, qui lui sourit tristement.

\— Joseph Rouletabille

Puisque Mme Mathieu a bien voulu avouer tout l’intérêt qu’elle portait au garde…

\— Le père Mathieu

La coquine !

\— Le président

Faites sortir le père Mathieu !

\——

On emmene le père Mathieu.

\— Joseph Rouletabille

Puisqu’elle a fait cet aveu, je puis bien vous dire qu’elle avait souvent des conversations, la nuit, avec le garde, au premier étage du donjon, dans la chambre qui fut, autrefois un oratoire. Ces conversations furent surtout fréquentes dans les derniers temps, quand le père Mathieu était cloué au lit par ses rhumatismes.

Une piqûre de morphine, administrée à propos, donnait au père Mathieu le calme et le repos, et tranquillisait son épouse pour les quelques heures pendant lesquelles elle était dans la nécessité de s’absenter. Mme Mathieu venait au château, la nuit, enveloppée dans un grand châle noir qui lui servait autant que possible à dissimuler sa personnalité et la faisait ressembler à un sombre fantôme qui, parfois, troubla les nuits du père Jacques. Pour prévenir son ami de sa présence, Mme Mathieu avait emprunté au chat de la mère Agenoux, une vieille sorcière de Sainte-Geneviève-des-Bois, son miaulement sinistre ; aussitôt, le garde descendait de son donjon et venait ouvrir la petite poterne à sa maîtresse.

\coupe

Quand les réparations du donjon furent récemment entreprises, les rendez-vous n’en eurent pas moins lieu dans l’ancienne chambre du garde, au donjon même, la nouvelle chambre, qu’on avait momentanément abandonnée à ce malheureux serviteur, à l’extrémité de l’aile droite du château, n’étant séparée du ménage du maître d’hôtel et de la cuisinière que par une trop mince cloison.

Mme Mathieu venait de quitter le garde en parfaite santé, quand le drame du « petit bout de cour » survint. Mme Mathieu et le garde, n’ayant plus rien à se dire, étaient sortis du donjon ensemble… Je n’ai appris ces détails, monsieur le président, que par l’examen auquel je me livrai des traces de pas dans la cour d’honneur, le lendemain matin… Bernier, le concierge, que j’avais placé, avec son fusil, en observation derrière le donjon, ainsi que je lui permettrai de vous l’expliquer lui-même, ne pouvait voir ce qui se passait dans la cour d’honneur. Il n’y arriva un peu plus tard qu’attiré par les coups de revolver, et tira à son tour. Voici donc le garde et Mme Mathieu, dans la nuit et le silence de la cour d’honneur. Ils se souhaitent le bonsoir ; Mme Mathieu se dirige vers la grille ouverte de cette cour, et lui s’en retourne se coucher dans sa petite pièce en encorbellement, à l’extrémité de l’aile droite du château.

\coupe

Il va atteindre sa porte, quand des coups de revolver retentissent ; il se retourne ; anxieux, il revient sur ses pas ; il va atteindre l’angle de l’aile droite du château quand une ombre bondit sur lui et le frappe. Il meurt. Son cadavre est ramassé tout de suite par des gens qui croient tenir l’assassin et qui n’emportent que l’assassiné. Pendant ce temps, que fait Mme Mathieu ? Surprise par les détonations et par l’envahissement de la cour, elle se fait la plus petite qu’elle peut dans la nuit et dans la cour d’honneur.

La cour est vaste, et, se trouvant près de la grille, Mme Mathieu pouvait passer inaperçue. Mais elle ne « passa » pas. Elle resta et vit emporter le cadavre. Le cœur serré d’une angoisse bien compréhensible et poussée par un tragique pressentiment, elle vint jusqu’au vestibule du château, jeta un regard sur l’escalier éclairé par le lumignon du père Jacques, l’escalier où l’on avait étendu le corps de son ami ; elle « vit » et s’enfuit. Avait-elle éveillé l’attention du père Jacques ? Toujours est-il que celui-ci rejoignit le fantôme noir, qui déjà lui avait fait passer quelques nuits blanches.

\coupe

Cette nuit même, avant le crime, il avait été réveillé par les cris de la « Bête du Bon Dieu » et avait aperçu, par sa fenêtre, le fantôme noir… Il s’était hâtivement vêtu et c’est ainsi que l’on s’explique qu’il arriva dans le vestibule, tout habillé, quand nous apportâmes le cadavre du garde. Donc, cette nuit-là, dans la cour d’honneur, il a voulu sans doute, une fois pour toutes, regarder de tout près la figure du fantôme. Il la reconnut. Le père Jacques est un vieil ami de Mme Mathieu. Elle dut lui avouer ses nocturnes entretiens, et le supplier de la sauver de ce moment difficile ! L’état de Mme Mathieu, qui venait de voir son ami mort, devait être pitoyable. Le père Jacques eut pitié et accompagna Mme Mathieu, à travers la chênaie, et hors du parc, par delà même les bords de l’étang, jusqu’à la route d’Épinay. Là, elle n’avait plus que quelques mètres à faire pour rentrer chez elle. Le père Jacques revint au château, et, se rendant compte de l’importance judiciaire qu’il y aurait pour la maîtresse du garde à ce qu’on ignorât sa présence au château, cette nuit-là, essaya autant que possible de nous cacher cet épisode dramatique d’une nuit qui, déjà, en comptait tant ! Je n’ai nul besoin de demander à Mme Mathieu et au père Jacques de corroborer ce récit.

« Je sais » que les choses se sont passées ainsi ! Je ferai simplement appel aux souvenirs de M. Larsan qui, lui, comprend déjà comment j’ai tout appris, car il m’a vu, le lendemain matin, penché sur une double piste où l’on rencontrait voyageant de compagnie, l’empreinte des pas du père Jacques et de ceux de madame.

\——

Ici, Rouletabille se tourne vers Mme Mathieu qui était restée à la barre, et lui fait un salut galant.

\transition{Fondu au noir.}

\Generique

Défilement des crédits

\end{document}

